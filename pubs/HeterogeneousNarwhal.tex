\documentclass{article}
\usepackage{markdown}
\usepackage{soulutf8}
%\renewcommand{\st}[1]{}
%\usepackage[scale=.75]{geometry}
\usepackage{placeins}
\usepackage{amsthm,amssymb,hyperref,marginnote}
\usepackage{subfig}
\usepackage{enotez}
\setenotez{counter-format=Alph}
\usepackage[textsize=footnotesize]{luatodonotes}%\usepackage{todonotes}%
\usepackage{xcolor}
\newcommand{\nb}[1]{%
  \todo[color=blue!60!black,shadow]{NB:\\ #1}%
}

\usepackage{newunicodechar}
% misc
\newunicodechar{∆}{\ensuremath{\Delta}}
\newunicodechar{⦄}{\ensuremath{\}}}
\newunicodechar{⦃}{\ensuremath{\{}}
\newunicodechar{¯}{\ensuremath{{}^{-}}}
\newunicodechar{·}{\ensuremath{\cdot}}
\newunicodechar{·}{\ensuremath{\cdot}}
\newunicodechar{∫}{\ensuremath{\int}}
\newunicodechar{□}{\Box}
\newunicodechar{⊗}{\ensuremath{\otimes}}
\newunicodechar{★}{\ensuremath{\star}}
\newunicodechar{∗}{^*}
\newunicodechar{♯}{\sharp}
\newunicodechar{①}{(1)}
\newunicodechar{②}{(2)}
\newunicodechar{☺}{\ensuremath{\smiley}}
\newunicodechar{⊣}{\ensuremath{\dashv}}

% advanced typography stuff
\newunicodechar{“}{``}
\newunicodechar{”}{''}
\newunicodechar{„}{\glqq}


\newunicodechar{fi}{fi}
\newunicodechar{ff}{ff}
\newunicodechar{fl}{fl}
\newunicodechar{ü}{{\"{u}}}
\newunicodechar{ä}{{\"{a}}}
\newunicodechar{ö}{{\"{o}}}
\newunicodechar{∖}{\ensuremath{\setminus}}

%math symbols
\newunicodechar{∃}{\exists}
\newunicodechar{∀}{\ensuremath{\forall}}
\newunicodechar{⇒}{\ensuremath{\Rightarrow}}
\newunicodechar{⇔}{\Leftrightarrow}

% white spaces, etc
%\DeclareUnicodeCharacter{00A0}{~}%\newunicodechar{ }{~}
\newunicodechar{␣}{\_}
\newunicodechar{…}{\ifmmode\dotsc\else\ldots\fi}
\newunicodechar{⋯}{\dotsm}
\newunicodechar{–}{\textendash}


% SUBSCRIPTS
%% letters
\newunicodechar{ₙ}{_n}
\newunicodechar{ₘ}{_m}
\newunicodechar{ₕ}{_h}
\newunicodechar{ᵢ}{_i}
\newunicodechar{ⱼ}{_j}
\newunicodechar{ₖ}{_k}
\newunicodechar{ₜ}{_t}
\newunicodechar{ₛ}{_s}
\newunicodechar{ᵤ}{_u}
\newunicodechar{ᵥ}{_v}
%% numbers
\newunicodechar{₀}{\ensuremath{{}_0}}
\newunicodechar{₁}{\ensuremath{{}_1}}
\newunicodechar{₂}{_2}
\newunicodechar{₆}{_6}

% superscripts
\newunicodechar{ⁿ}{^n}
\newunicodechar{ᵐ}{^m}
\newunicodechar{ˢ}{^s}
\newunicodechar{ᵗ}{\ensuremath{{}^t}}
\newunicodechar{ⁱ}{^i}
\newunicodechar{ʲ}{^j}
\newunicodechar{ᵏ}{^k}
\newunicodechar{ᵗ}{^t}
\newunicodechar{¹}{\ensuremath{{}^1}}
\newunicodechar{²}{^2}
\newunicodechar{³}{^3}
\newunicodechar{ᵀ}{^{\rm T}}


% constants and scalarlike
\newunicodechar{∞}{\infty}

% math operators
%\DeclareUnicodeCharacter{00D7}{\ensuremath{\times}}%\newunicodechar{×}{\times}
\newunicodechar{÷}{\div}
%\DeclareUnicodeCharacter{2190}{\ensuremath\leftarrow}%\newunicodechar{←}{\ensuremath\leftarrow}
%\DeclareUnicodeCharacter{2192}{\ensuremath\rightarrow}%\newunicodechar{→}{\ensuremath\rightarrow}
%\DeclareUnicodeCharacter{21E2}{\ensuremath\dashrightarrow}%\newunicodechar{⇢}{\ensuremath\dashrightarrow}
%\DeclareUnicodeCharacter{21A6}{\ensuremath\mapsto}%\newunicodechar{↦}{\ensuremath\mapsto}
%\DeclareUnicodeCharacter{2196}{\ensuremath\nwarrow}%↖
%\DeclareUnicodeCharacter{2197}{\ensuremath\nearrow}%↗
%\DeclareUnicodeCharacter{2198}{\ensuremath\searrow}%↘
%\DeclareUnicodeCharacter{2199}{\ensuremath\swarrow}%↙
        
% sets 
\newunicodechar{∅}{\varnothing}

% analysis
\newunicodechar{∫}{\int}
\newunicodechar{ℓ}{\ell}

% math relation symbols
\newunicodechar{∈}{\ensuremath{\in}}
\newunicodechar{∉}{\notin}
\newunicodechar{∋}{\ni}
\newunicodechar{≅}{\cong}
\newunicodechar{≥}{\geq}
\newunicodechar{≤}{\leq}
\newunicodechar{≪}{\ll}
\newunicodechar{≫}{\rr}
\newunicodechar{⋘}{\lll}
\newunicodechar{⋙}{\rrr}
%\DeclareUnicodeCharacter{2260}{\neq}%\newunicodechar{≠}{\neq}
\newunicodechar{⊂}{\subset}
\newunicodechar{⊆}{\ensuremath{\subseteq}}
\newunicodechar{⊃}{\supset}
\newunicodechar{⊇}{\supseteq}
\newunicodechar{⊑}{\sqsubseteq}
\newunicodechar{⊒}{\sqsupseteq}

% binary operators
\newunicodechar{⋃}{\bigcup}
\newunicodechar{∪}{\ensuremath{\cup}}
\newunicodechar{⋂}{\bigcap}
\newunicodechar{∩}{\cap}
\newunicodechar{∘}{\circ}

% math arrows 
\newunicodechar{↑}{\ensuremath{\uparrow}}
\newunicodechar{↓}{\ensuremath{\downarrow}}

\newunicodechar{⇀}{\ensuremath{\rightharpoonup}}
\newunicodechar{✉}{\ensuremath{\Letter}}

% summation/products etc (operator for families)
%\DeclareUnicodeCharacter{2211}{\sum}%\newunicodechar{∑}{\sum}
%\DeclareUnicodeCharacter{220F}{\prod}%\newunicodechar{∏}{\prod}

% greek lower case letters (math)
\newunicodechar{α}{\ensuremath{\alpha}}
\newunicodechar{β}{\ensuremath{\beta}}
\newunicodechar{γ}{\ensuremath{\gamma}}
\newunicodechar{δ}{\ensuremath{\delta}}
\newunicodechar{ε}{\ensuremath{\varepsilon}}
\newunicodechar{ϵ}{\ensuremath{\epsilon}}
%\newunicodechar{ϵ}{\varepsilon}
\newunicodechar{η}{\ensuremath{\eta}}
\newunicodechar{λ}{\ensuremath{\lambda}}
\newunicodechar{κ}{\ensuremath{\kappa}}
\newunicodechar{μ}{\ensuremath{\mu}}
\newunicodechar{µ}{\ensuremath{\mu}}
\newunicodechar{ν}{\ensuremath{\nu}}
\newunicodechar{ρ}{\ensuremath{\rho}}
\newunicodechar{σ}{\ensuremath{\sigma}}
\newunicodechar{ξ}{\ensuremath{\xi}}
\newunicodechar{π}{\ensuremath{\pi}}
\newunicodechar{ω}{\ensuremath\omega}
\newunicodechar{ϖ}{\ensuremath{\varpi}}
\newunicodechar{θ}{\ensuremath{\theta}}
\newunicodechar{φ}{\ensuremath{\phi}}
\newunicodechar{ψ}{\ensuremath{\psi}}


% greek upper case letters (math)
\newunicodechar{Θ}{\Theta}
\newunicodechar{Φ}{\ensuremath{\Phi}}
\newunicodechar{Ω}{\Omega}
\newunicodechar{Δ}{\Delta}
\newunicodechar{Σ}{\ensuremath{\Sigma}}
\newunicodechar{Γ}{\ensuremath{\Gamma}}


% calligraphic upper case
\newunicodechar{𝓐}{\ensuremath{\mathcal{A}}}
\newunicodechar{𝓑}{\ensuremath{\mathcal{B}}}
\newunicodechar{𝓒}{\ensuremath{\mathcal{C}}}
\newunicodechar{𝓓}{\ensuremath{\mathcal{D}}}
\newunicodechar{𝓔}{\ensuremath{\mathcal{E}}}
\newunicodechar{𝓕}{\ensuremath{\mathcal{F}}}
\newunicodechar{𝓚}{\ensuremath{\mathcal{K}}}
\newunicodechar{𝓛}{\ensuremath{\mathcal{L}}}
\newunicodechar{𝓝}{\ensuremath{\mathcal{N}}}
\newunicodechar{𝓡}{\ensuremath{\mathcal{R}}}
\newunicodechar{𝓢}{\ensuremath{\mathcal{S}}}
\newunicodechar{𝓧}{\ensuremath{\mathcal{X}}}
\newunicodechar{𝓨}{\ensuremath{\mathcal{Y}}}
\newunicodechar{𝓩}{\ensuremath{\mathcal{Z}}}
\newunicodechar{𝓥}{\ensuremath{\mathcal{V}}}

% fraktur
\newunicodechar{𝕼}{\mathfrak{Q}}

% "normal" boldface
\newunicodechar{𝕍}{\ensuremath{\mathbb{V}}}
\newunicodechar{ℂ}{\ensuremath{\mathbb{C}}}
\newunicodechar{𝔻}{\ensuremath{\mathbb{D}}}
\newunicodechar{𝔼}{\ensuremath{\mathbb{E}}}
\newunicodechar{𝔽}{\ensuremath{\mathbb{F}}}
\newunicodechar{𝕂}{\ensuremath{\mathbb{K}}}
\newunicodechar{ℙ}{\ensuremath{\mathbb{P}}}
\newunicodechar{ℚ}{\ensuremath{\mathbb{Q}}}
\newunicodechar{ℕ}{\ensuremath{\mathbb{N}}}
\newunicodechar{𝕄}{\ensuremath{\mathbb{M}}}
\newunicodechar{ℝ}{\ensuremath{\mathbb{R}}}
\newunicodechar{𝕊}{\ensuremath{\mathbb{S}}}
\newunicodechar{𝕋}{\ensuremath{\mathbb{T}}}
\newunicodechar{ℤ}{\ensuremath{\mathbb{Z}}}
\newunicodechar{𝟙}{\ensuremath{\mathbb{1}}}
\newunicodechar{⨾}{\fatsemi}

\newunicodechar{⟨}{\ensuremath{\langle}}
\newunicodechar{⟩}{\ensuremath{\rangle}}
\newunicodechar{⟪}{\ensuremath{\langle\!\langle}}
\newunicodechar{⟫}{\ensuremath{\rangle\!\rangle}}
\newunicodechar{⌈}{\ensuremath{\lceil}}
\newunicodechar{⌉}{\ensuremath{\rceil}}
\newunicodechar{⌊}{\ensuremath{\lfloor}}
\newunicodechar{⌋}{\ensuremath{\rfloor}}


\newunicodechar{½}{\ensuremath{\sfrac{1}{2}}}
\newunicodechar{ʏ}{\textsc{y}}
\newunicodechar{ᴘ}{\textsc{p}}
\newunicodechar{ʜ}{\textsc{h}}
\newunicodechar{ᴏ}{\textsc{o}}
\newunicodechar{ɴ}{\textsc{n}}

\newunicodechar{‽}{\textinterrobang}
\newunicodechar{⁈}{\textinterrobang}
\newunicodechar{‼}{{\bf \color{red}{!!}}}


%%% Local Variables:
%%% mode: latex
%%% TeX-engine: luatex
%%% TeX-command-extra-options: "-shell-escape"
%%% TeX-master: "HeterogeneousNarwhal"
%%% End:

\theoremstyle{definition}
\newtheorem{definition}{Definition}
\usepackage{amsmath}
% \usepackage[T1]{fontenc}
% \usepackage{lmodern}
%\usepackage[utf8]{inputenc}
\usepackage{xspace}
% macros
\newcommand{\tnote}[1]{
  \marginnote{\footnotesize #1}%
}
\newcommand{\rtnote}[1]{%
  \reversemarginpar%
  \tnote{#1}%
  \normalmarginpar%
}

\newcommand{\base}[1][ ]{%
  base ledger%
  \ifthenelse{\equal{#1}{ }}{}{#1}
}
\newcommand{\Base}[1][ ]{%
  Base ledger
  \ifthenelse{\equal{#1}{ }}{}{#1}
}

% \dag is defined to produce † unfortunately
\newcommand{\DAG}[1][]{\textsc{Dag}#1\xspace}
\newcommand{\Dag}[1][]{\textsc{dag}#1\xspace}

\newcommand{\fifo}{\textsc{fifo}}
\newcommand{\Fifo}{\textsc{Fifo}}
\newcommand{\aka}[1][]{a.k.a.\xspace}
\newcommand{\ie}[1][]{\emph{i.e.}, }
\newcommand{\eg}[1][]{\emph{e.g.}, }
\newcommand{\fig}[1][]{Fig.~}
\newcommand{\Learner}{%
  % the set of learners
  \ensuremath{L}
}
\newcommand{\Q}[1]{%
  % trust live
  Q_{#1}%
}
\newcommand{\rough}[1][ ]{%
  %\fbox{\color{blue!70!black}!!}%
  \ifthenelse{\equal{#1}{ }}%
  {{\color{red}{\bf!!}}}%
  {{\color{blue!70!black}\ul{#1}}}%
}
\newcommand{\circledX}[1]{\tikz[baseline={(x.south)}]{\node[circle,draw,inner sep=.3pt,outer sep=0pt,very thin](x){\tiny #1};}}
\usepackage{newunicodechar}
%\newunicodechar{}{\ensuremath{}}  
\newunicodechar{₁}{\ensuremath{{}_1}}
\newunicodechar{₂}{\ensuremath{{}_2}}
\newunicodechar{₃}{\ensuremath{{}_3}}
\newunicodechar{₄}{\ensuremath{{}_4}}
\newunicodechar{₅}{\ensuremath{{}_5}}
\newunicodechar{★}{\ensuremath{*}~}
\newunicodechar{ }{~}
%\newunicodechar{①}{\circledX 1}
\newunicodechar{“}{``}
\newunicodechar{”}{''}
%\newunicodechar{②}{\circledX 2}
\newunicodechar{ₐ}{\ensuremath{{}_a}}
\newunicodechar{ₚ}{\ensuremath{{}_p}}
\newunicodechar{‼}{\rough}
\newunicodechar{‽}{\ensuremath{?!}}
\newunicodechar{↑}{\ensuremath{\uparrow}}  
\newunicodechar{⇑}{\ensuremath{\Uparrow}}  
\newunicodechar{♯}{\ensuremath{\sharp%\hat\#
}}  
\newunicodechar{∅}{\ensuremath{\varnothing}}
\newunicodechar{≠}{\ensuremath{\neq}}              
\newunicodechar{∩}{\ensuremath{\cap}}              
\newunicodechar{≡}{\ensuremath{\equiv}}              
\newunicodechar{∈}{\ensuremath{\in}}
\newunicodechar{ℝ}{\ensuremath{\mathbb{R}}}
\newunicodechar{↔}{\ensuremath{\leftrightarrows}}    
\newunicodechar{→}{\ensuremath{\rightarrow}}
\newunicodechar{←}{\ensuremath{\leftarrow}}
\newunicodechar{⇒}{\ensuremath{\Rightarrow}}
\newunicodechar{∀}{\ensuremath{\forall}}
\newunicodechar{‌}{\allowbreak }
\newunicodechar{‍}{{}}                                

\usepackage{tikzpeople}%‼ for evil validators etc. 

\usepackage{tikz}
\usetikzlibrary{shapes,positioning,fit,backgrounds}
\usetikzlibrary{patterns,intersections,calc}
\newcommand{\qs}[1][~]{\tikz[baseline={([yshift=0pt]theNode.base)}]{\node[rectangle,
,double,inner sep=.5pt,outer sep=0pt,fill=black] (theNode){\textcolor{white}{\footnotesize \bf q#1}};}
}
\newcommand{\bk}[1][green!60!black]{\tikz[baseline={([yshift=0pt]theNode.base)}]{\node[regular polygon, regular polygon sides=6
,double,inner sep=.5pt,outer sep=0pt,fill=#1] (theNode){\textcolor{white}{\footnotesize \bf bk}};}}
\newcommand{\ac}{\tikz[baseline={([yshift=0pt]theNode.base)}]{\node[regular polygon, regular polygon sides=6
,double,inner sep=.5pt,outer sep=0pt,fill=black] (theNode){\textcolor{white}{\footnotesize \bf a}};}}

\newcommand{\hd}[1][ ]{%
  \ifthenelse{\equal{#1}{}}%
  {\tikz[baseline={([yshift=0pt]theNode.base)}]{
      \node[rectangle,inner sep=1.5pt,outer sep=0pt,double] (theNode){\textcolor{black}{\footnotesize \bf \ul{HD}}};
    }}%
  {\tikz[baseline={([yshift=0pt]theNode.base)}]{
      \node[rectangle,double,inner sep=1.5pt,outer sep=0pt,double,draw] (theNode){\textcolor{black}{\footnotesize \bf HD}};
      
    }}%
}
\newcommand{\wh}[1][ ]{%
  \tikz[baseline={([yshift=0pt]theNode.base)}]{%
    \ifthenelse{\equal{#1}{ }}%
    {\node[rectangle,fill=black,inner sep=1.5pt,outer sep=0pt] (theNode){\textcolor{white}{\footnotesize \bf WH}};}%
    {\node[rectangle,draw,fill=lightgray,inner sep=1.5pt,outer sep=0pt] (theNode){\textcolor{black}{\footnotesize  WH}};}%
  }%
}

\newcommand{\anItemInline}[6][theNode]{%
  % #1 the name of the node (just in case, remember picture is on)
  % #2 shape (e.g., ellipse)
  % #3 fill color (e.g., black)
  % #4 draw color (e.g., green, or none)
  % #5 text color (e.g., white)
  % #6 the actual text (e.g., \bf TX)
  \tikz[baseline={([yshift=0pt]#1.base)},remember picture]{%
    \node[#2,fill=#3,draw=#4,inner sep=.5pt,outer sep=0pt] %
    (#1){%
      \textcolor{#5}{%
        \footnotesize #6%
      }%
    };%
  }%
  % additional “decorations” via another pic, 
  % (with `remember picture` and `overlay`)
}

\newcommand{\tx}[1][theTX]{%
  \anItemInline[#1]%
  {ellipse}%
  {black}%
  {none}%
  {white}%
  {\bf TX}%
}
% \newcommand{\tx}[1][]{%
%   \tikz[baseline={([yshift=0pt]theNode.base)}]{%
%     \node[ellipse,fill=black,inner sep=.5pt,outer sep=0pt] %
%     (theNode){%
%       \textcolor{white}{\footnotesize \bf TX\makebox[0pt][l]{\ensuremath{{}_{#1}}}}};%
%   }%
% }
\newcommand{\es}{\tikz[baseline={([yshift=0pt]theNode.base)}]{\node[ellipse,fill=white,draw,thick,inner sep=.5pt,outer sep=0pt] (theNode){\textcolor{black}{\footnotesize TX}};}}
\newcommand{\rnd}{\ensuremath{\mathrm{rnd}}}
\newcommand{\cnt}{\ensuremath{\mathrm{cnt}}}
% \usepackage{ebgaramond}
% \usepackage[cmintegrals,cmbraces]{newtxmath}
% \usepackage{ebgaramond-maths}
% \usepackage{amssymb}
\usepackage{fontspec}
% \setmainfont{Asana-Math}

%https://tex.stackexchange.com/questions/425098/which-opentype-math-fonts-are-available
\usepackage{comment}
%\includecomment{comment}
\let\oldendnote\endnote
\renewcommand{\endnote}[2][]{%
  \marginnote{\oldendnote[#1]{#2}}%
}
\title{%
  Heterogeneous Narwhal: % 
  \\
  a cross-chain mempool \Dag%
}
\author{Typhon Team Heliax}
\date{\today}
\begin{document}
%\renewcommand{\todo}[2][]{}
%\renewcommand{\endnote}[1]{#1}
%\begin{comment}
\maketitle

\begin{abstract}
  \noindent
  Blockchains exhibit \emph{linear} structure, 
  resulting from a \emph{single} reference to the previous block.
  If instead, 
  each block may reference \emph{several} previous blocks,
  we can build a \emph{directed acyclic graph} (\Dag) of blocks.
  In fact,
  such “block \Dag[s]” are the basic data structure that
  several recent consensus algorithms rely on,
  \eg \Dag-rider, Bullshark, and Narwhal\&Tusk.\footnote{%
    The respective references are, 
  }
  These protocols build 
  a growing “global” \Dag of transaction data such that%
  ---among other things---%
  ① every validator's local view is a sub-\Dag of the “global” \Dag and %
  ② every node of the “global” \Dag is
  “logged”\endnote{%
    Logging is terminology borrowed from César Sanchez:
    he likes to talk about \emph{log-chain}s,
    or more verbose “mempool \emph{log system}”
    cite \href{https://arxiv.org/abs/2206.11845}{%
    Setchain: Improving Blockchain Scalability
    with Byzantine Distributed Sets and Barriers}
  }
  for inclusion into a total order of blocks. 
  Such “global” \Dag[s] will be the topic of the present paper, 
  referred to as {Mempool \Dag[s]}, or \emph{mem-\slshape\Dag[s]}, 
  for short.  

  The paper introduces a cross-ledger mem-\Dag, 
  generalizing Narwhal's single ledger mem-\Dag;
  in analogy to Heterogeneous Paxos,
  we dub it \emph{Heterogenous Narwhal}.   
  Heterogeneous Narwhal\&Paxos are an alternative for bridges. 
  \todo{%
    we actually need to explain how heterogeneous Narwhal is 
    1. an alternative to bridges
    2. which benefits / drawback we have, in general
    3. which parts are “implementation detail”.
  }
  We also describe how being scheduled for inclusion
  should lead to eventual inclusion,
  and which hurdles need to be taken to achieve this. 
\end{abstract}

\tableofcontents
\section{Introduction}

\todo[inline]{main contribs:\\[\baselineskip]
  \begin{minipage}{1.0\linewidth}
    - alternative to bridges (between very different chains ?)\\
    - merging might be actually a very good thing (for very small chains)\\ 
    - chimera chains (between big chains)
  \end{minipage}\\[\baselineskip]
  In summary,
  we present a general framwork for building
  a cross-chain ecosystem of pre-existing and new ledgers.  
}


\section{Motivation / Background / Context}

\subsection{Atomic cross-chain swaps}
\label{sec:atom-trans-across}

\begin{quote}
  An atomic cross-chain swap is 
  a distributed coordination task 
  where multiple parties exchange assets across multiple blockchains, 
  for example, trading bitcoin for ether.

  An atomic swap protocol guarantees 
  (1) if all parties conform to the protocol, 
  then all swaps take place, 
  (2) if some coalition deviates from the protocol, 
  then no conforming party ends up worse off, and 
  (3) no coalition has an incentive to deviate from the protocol.
\hfill\cite{podc18Herlihy} 
\end{quote}

\subsection{Let's start in the land of naïveté}
\label{sec:naivities}

\subsubsection{The impossible}
\label{sec:impossible-concurrency}

One simple ``ideal'' solution for cross-chain swaps is 
``concurrent''  inclusion of 
transactions in two blocks, one on each chain, 
which refer to each other reciprocally.
However, 
this simply cannot be accomplished with hash links.

\subsubsection{A seemingly good situation}
\label{sec:almost-safe}

Another naïve approach consists in
% in a \emph{combo block}  
% is proposing the next ``same'' block on two chains,
% at the same “moment”,
% e.g., if on 
is running a validator on both involved chains. 
Whenever the validator has the right to propose 
the next block on both chains,
we “simultaneously” propose one transaction on each chain. 
However,
we could “loose” the proposal spot,
\eg if a view change is initiated on one of the two chains.
\todo{%
  describe in more detail 
}
Even worse,
it could be that one of the proposals goes through,
while the other gets “vetoed”.

\subsubsection{The envisioned solution: chimera chains}
\label{sec:envisioned-solution}
We propose to establish a ledger that records transactions
that involve two (or more) pre-existing chains%
---dubbed \emph{chimera chain}. 
If we want to speak in terms of bridges,
our envisioned solution combines the ideas of trusted and trustless bridges.
 \todo[inline]{%
  add fine print on which bridges might actually do this already:
  in particular, trustless bridges,
  see \eg \url{https://ethereum.org/en/bridges/}
}

The chimera chains orders blocks with cross-chain transactions
such that inclusion in 
the chimera chain leads to (eventual) execution in the two \base[s]. 
The seemingly simple principle for running the chimera chain
is maximal usage of pre-existing validator sets.
\endnote{%
  though additional validators might want to join for 
  the purpose of bridging for the purpose of chimera chains
}
In particular, we want
\begin{enumerate}
\item more than a single validator in the intersection of two quorums
\item a protocol, 
  that gives these validators the opportunity to propose \emph{combo block}
  on a chimera chain. 
\end{enumerate}
\todo{
  explain in non-technical terms that
  the same-slot-on-both-chains situation
  has a very intuitive counterpart in
  having quorum-pairs between 
  two chains as the quorum system for the chimera chain 
  of two chains. 
}\endnote{%
  for “quorum-pairs” between two chains,
  explain why it is still much cheaper
  than having a merged super-chain
  (if it is ...)
}


\subsection{Chimera chains}
\label{sec:chimera-chains}

% \paragraph{Overview}

Assuming a fixed set of \base[s], 
each of which have a set of quorums
such that one of them “should” be live at any point in time,%
\footnote{%
  This is a simplifying assumption,
  which we use only in the exposition.
  In principle,
  learners are completely independent entities.
  They “dictate” which quorums they would trust,  
  in particular concerning liveness.
  For safety, %
  the situation is more intricate. %
}%
\todo{explain this}\xspace%
chimera chains have the following phases:
\begin{description}
\item[Spawn] %
  Among other things,
  the genesis blocks are generated,
  including links to the blocks in all \base[s]
  that “summoned” the chimera chain. 

\item[Active]
  The usual mode of operation,
  in particular facilitating cross-chain transfers. 

\item[Retire] 
  Low activity or inconsistencies might call 
  for retirement of the chimera chain. 

\item[Merge]
  In principle,
  the ecosystems of two \base[s] do “completely” merge
  such that the chimera chain is still performant enough.
  In fact,
  this might be a good thing:
  smaller \base[s] are just joining larger ones.
  Clearly,
  there's a trade off,
  between centralization and usability. 
\end{description}

The chimera chain has dedicated “administrative” transactions,
e.g.,
for spawing











\section{Overview: %
  mem-DAGs, heterogeneity, etc.}
\label{sec:overview}

Recall that mem-\Dag[s] build a \Dag of blocks,
each referencing a quorum of previous blocks;
a block in the \Dag can be \emph{committed} 
(using virtually any consensus algorithm)
if it is referenced by a weak quorum (from the next “layer” in the \Dag). 
An example of such a \Dag
is shown in \fig\ref{fig:blue_dag}. 
In the present paper,
we describe a protocol for concurrent building 
of entangled \Dag[s] for a whole ecosystem of different ledgers,
in the spirit of a cross-chain world. 

\begin{figure}[htb]
  \centering
  \includegraphics[width=.95\linewidth]{./blue_dag.png}
  \caption{A mem-\Dag (self-references omitted)}
  \label{fig:blue_dag}
\end{figure}


\begin{figure}[htb]
  \centering
  
  \begin{tikzpicture}
    \node[draw] (avl) {transaction data};

    \node[draw,right=1ex of avl] (dag) {``causal'' \Dag structure};
    \begin{pgfonlayer}{background}
    \node (x) [fill=lightgray,fit=(avl)(dag)] {};
  \end{pgfonlayer}
  \node[above=2ex of x,draw,double] {uniqueness of headers};
\end{tikzpicture}\protect\todo{where to put the signed quorums ?!}
  \caption%
  [Interdependence of availability and integrity]%
  {Illustration of 
    the interdependence of the availability and 
    integrity protocols}
  \label{fig:availability-n-integrity}
\end{figure}



Roughly, we have two complementary protocols running concurrently: 
 \begin{enumerate}
 \item the availability protocol; and
 \item the integrity protocol. 
 \end{enumerate}
 The availability protocol makes sure
 that transaction data is available 
 as long as necessary;\footnote{%
   There is some fine print concerning 
   the conditions under which this is actually the case. 
 }
 moreover,
 the availability protocol 
 is tasked with keeping available 
 the \emph{signed headers},
 \todo[inline]{explain \texttt{signed headers}}
 which the integrity protocol produces. 

 The integrity protocol makes sure that 
 each validator can only produce one block in each of its (local) rounds. 


 

\section{Architecture and communication patterns}
\label{sec:communication-patterns}
 We incorporate Narwhal's~\cite{NT} 
 scale out architecture%in our availability protocol 
 : %
 each validator has a unique \emph{primary} and % 
 a number of \emph{workers} % 
 (see \fig\ref{fig:validators})%
 . 
\begin{figure}[htb]
  \centering

validator:\(\left\{\begin{tikzpicture}[baseline={(p.south)},remember picture]
    \node[circle,draw,inner sep=1.5ex] (p) at (5.5,1.2) {p};
    \foreach \i in {1,...,10}{
      \coordinate (p_\i) at  (p.160+20*\i);
    }
    \foreach \i in {1,...,10}{
      \node[rectangle,fill=lightgray](w_\i) at (\i,0){\(w_{\i}\)};
      \draw[->] (w_\i.north) .. controls +(0,.2) .. (p_\i);
    }
    \begin{pgfonlayer}{background}
      \node[rounded corners,fill=blue!30!white,inner sep=2ex] (background) [fit=(w_1.south west)(w_10.south east)(p)] {};
    \end{pgfonlayer}
  \end{tikzpicture}\right.\)

\vspace{4ex}

\tikz[remember picture]{\node[rectangle,fill=gray,draw] (c) {client};}\hfill~\\

\vspace{4ex}

validator:\(\left\{\begin{tikzpicture}[baseline={(q.north)},remember picture,yscale=-1]
    \node[circle,draw,inner sep=1.5ex] (q) at (5.5,1.2) {q};
    \foreach \i in {1,...,10}{
      \coordinate (q_\i) at  (q.200-20*\i);
    }
    \foreach \i in {1,...,10}{
      \node[rectangle,fill=lightgray](v_\i) at (\i,0){\(v_{\i}\)};
      \draw[->] (v_\i.south) .. controls +(0,.2) .. (q_\i);
    }
    \begin{pgfonlayer}{background}
      \node[rounded corners,fill=blue!30!white,inner sep=2ex] (background) [fit=(v_1.north west)(v_10.north east)(q)] {};
    \end{pgfonlayer}
    % --------------------------------------------------------------------------------
    \begin{scope}[overlay,thick]
          \foreach \j in {1,...,10}
    \draw[<->] (v_\j) -- (w_\j);

    \draw[<->] (c.-15) .. controls +(2,0) and +(-1,-1) .. (v_3.north west);
    \draw[<->] (c.15) .. controls +(2,0) and +(-1,1).. (w_7.south west);
    \draw[<->] (p) -- (q);
    \end{scope}
  \end{tikzpicture}\right.\)
  \caption{The structure and communication patterns of validators}
  \label{fig:validators}
\end{figure}



\section{Preliminaries}
\label{sec:preliminaries}
\todo[inline]{%
  ``copy'' relevant parts of % 
  the heterogeneous paxos tech report%
}

Let us fix an arbitrary learner graph. 

\begin{definition}[Global Weak Quorum]
  \label{def:global-weak-quorum}
  A %
  \emph{global weak quorum} %
  is a set~\(X\) that is a weak quorum for each learner,  %
  i.e., \(X ∩ qₐ ≠ ∅\) %
  for every learner~\(a ∈ \Learner\), and %
  all \(qₐ ∈ \Q{a}\).
\end{definition}

\begin{definition}[Universal Quorum]
  \label{def:universal-quorum}
  A universal quorum is a set
  that contains a qourum for each learner. 
\end{definition}
\todo{%
“upward” closure for live quorums seems a “wrong” assumption
}


\section{Worker actions}
\label{sec:worker-actions}
Every validator has the same number of workers. 
Thus,
each worker can be assigned a unique \emph{mirror worker} 
on every other validator.
We shall adopt the convention of using the same subscript
for mirror workers of each other. % 
Thus,
in \fig\ref{fig:validators},
workers~\(w_2\) and~\(v_2\) are mirror workers of each other.
In the integrity protocol,
workers are only tasked with “bookkeeping” matters; 
in particular,
they keep track of (batches of) transactions,
their hashes,
and erasure coding shares;
they only pass on hashed data and 
block header information to their primaries.
The idea is to keep the network bandwidth usage of primaries 
as low as possible. 
For example,
primaries do not need to send messages back to their workers. 

\subsection{The pure availability protocol: worker actions}
\label{sec:base-protocol}
\todo[inline]{get rid of the broadcast of the availability certificates}
\todo[inline]{%
  maybe we need to ``announce'' headers, % 
  to avoid ``crossing'' of round numbers of learner%
}
\todo[inline]{%
  so, signed qourums are an ``output'' of the integrity protocol,
  and need to be made available (such that it is possible to create new headers). 
}

\todo[inline]{%
  worker hashes also should include an availability certificate %
  of the header creator's previous header %
}

\begin{description}
\item[Transaction collection (\tx{}←)] 
  \tnote{worker\\ ← client}
  Each worker keeps listening 
  for new transaction requests from clients.\footnote{%
    The bandwith and amount of storage for storing incoming transactions
    \emph{should} be big enough 
    to process all incoming transactions. 
    We share this assumption with Byzantine set consensus \cite{RedBelly}. 
    Transaction fees are one way to avoid flooding attacks,
    making the latter prohibitively expensive. 
    For example,
    we might assume a \fifo-buffer
    although it is more likely
    a priority queue,
    based on a combination of fees and
    quality of service considerations. 
  }
  Transactions should be buffered using reasonably fast memory. 
  \endnote{%
    This matter should be discussed
    in the context of \textsc{p2p}. 
  }

\item[Transaction distribution (\tx⇒)]
  \tnote{worker\\ ⇒ worker}
  Each worker “broadcasts” erasure coding shares of 
  every received transaction to mirror workers.
  In the simplest case\xspace% 
  ---the one we cover first---%
  this amounts to broadcasting the transaction.\footnote{ 
  However, 
  if we perform proper erasure coding,  
  it is not broadcasting in the strict sense:
  each mirror worker receives a different message. 
  We first cover the case of trivial erasure coding
  ‼[covering the case of proper erasure codining in ‽].
  }
  In general,
  transaction distribution can be divided into two steps.
  \todo[inline]{
    we might require a 
    \emph{position number} for
    transactions of the current batch, 
    and a sequence number (of the block to be produced)
  }
  \begin{description}
  \item[Erasure coding via copying (\es)]
    \tnote{[worker]}
    We first cover the case of trivial erasure coding, 
    in line with the description of the homogeneous case~\cite{NT}. %
    Thus, erasure coding shares of a transactions are simply copies. 
    We visually distinguish between “copies” of transactions
    from the “original” transaction supplied by the client,  
    using the symbols \tx\ and \es, respectively. 
%%%%%%%%%%%%%%%%%%%%%%%%%%%%%%%%%%%%%%%%%%%%%%%%%%%%%%%%%%%%%%%%%%%%%%%%%%%%%%%%%%%%%%%%%
    \todo[inline]{put fwd ref / future work }
%%%%%%%%%%%%%%%%%%%%%%%%%%%%%%%%%%%%%%%%%%%%%%%%%%%%%%%%%%%%%%%%%%%%%%%%%%%%%%%%%%%%%%%%%    
  \item[Copy distribution (\es⇒)] 
    \tnote{worker\\ ⇒ worker }
    Each share of the erasure code,
    \ie a copy of the transaction, 
    is sent to every mirror worker.%
    \endnote{
      elaborate on the distribution scheme,
      and how the destination of each earasure share is determined
      (for the general erasure coding).
    }%
    \xspace
    We tag each transaction with a \emph{sequence number},
    consisting of the validator round and 
    the position in the list of transactions for 
    the next block (header) in which the transaction will be included. 
    % so that it becomes clear in which order the 
    % transactions of the next block (header) will be listed. 
  \end{description}
  Note that copies of transactions are \emph{not} signed by the worker. 
  Signatures are deferred to until after the last transaction  
  of a validator round, 
  when the worker will sign the hash of the list of all broadcast transactions,
  called a \emph{worker hash}. 
  
\item[Worker hash compilation (\wh)]
  \tnote{(worker)}
  Towards the end of a “validator round”,\endnote{%
    Is there any such thing as “validator round”?\\
    - sequence number\\
    - validator height\\
    - ... 
  }%
  \xspace
  each worker produces its worker hash for the broadcast batch of transactions. 
  In detail,
  a worker hash consists of 
  \begin{itemize}
  \item the hash of the broadcast list of transactions,
  \item the number of transactions, and
  \item the round number,  
  \end{itemize}
  signed by the worker. 
  \endnote{\color{red}
    For general erasure coding:
    \em Does this need to includes for each receiving worker,
    the (hashes of the) erasure coding shares that 
    they should have available. ?
  }
\item[Worker hash broadcast (\wh⇒)] 
  \tnote{worker\\ ⇒ worker }  
  A worker broadcasts its most recent worker hash
  to mirror workers.
\item[Worker hash provision (\wh↑)]
  \tnote{worker\\ → primary}
  The most recent worker hash is sent to the primary 
  for inclusion into the next header. 
  \todo{
    How much “additional” information do we have 
    to include into the header
    such that signing a header certificate becomes meaningful?
  }
  
\item[Worker hash reception and checking ({\wh[]←})] 
  \tnote{worker\\ ← worker}  
  At any time, 
  a worker can receive a worker hash 
  from a mirror worker. 
  As a first reaction, 
  it checks whether enough 
  transactions have been received by the worker and, 
  if so, 
  whether the hash of the list of transaction matches the worker hash. 

\item[Worker hash forward ({\wh[]⇑})]
  \tnote{worker\\ → primary}  
  If a worker has successfully checked 
  the availability of the transactions of a received worker hash, %
  it “forwards” the worker hash to its primary.\footnote{%
    Validators will use this information 
    to send availability commitments to block headers of other primaries.
  }
\end{description}



\begin{figure}[htb]
  \centering
  \tikzstyle{every node}+=[outer sep=0pt,inner sep=1pt]
  \newcommand{\primaryDistance}{15ex}
  \newcommand{\workerPrimaryDistance}{1ex}
  \newcommand{\workerDistance}{3ex}
  \scalebox{.9}{%
  \footnotesize%
  \begin{tikzpicture}[scale=1.2]
    %%%%%%%%%%%%%%%%%%%%%%%%%%%%%%%%%%%%%%%%%%%%%%%%%%%%%%%%%%%%%%%%%%%%%%%%%%%%%%%% 
    % The message passing diagram of the availability protocol at genesis
    %%%%%%%%%%%%%%%%%%%%%%%%%%%%%%%%%%%%%%%%%%%%%%%%%%%%%%%%%%%%%%%%%%%%%%%%%%%%%%%%
    % first the time lines for primaries and their workers
    \coordinate (primaryAnchor) at (0,0);
    \foreach \p in {1,...,5} {
      \node[below=\primaryDistance of primaryAnchor,anchor=east] (p\p) 
      at (primaryAnchor) {\ensuremath{\text{primary}_\p}};
      \draw[->] (p\p) -- ++(10.4,0);
      \coordinate (workerAnchor) at ([yshift=-\workerPrimaryDistance]p\p.east);
      \foreach \j in {1,...,2} {
        \node[below=\workerDistance of primaryAnchor,anchor=east] (w\p_\j) 
        at (workerAnchor) {\ensuremath{\text{worker}_{\p,\j}}};
        \draw[->] (w\p_\j) -- ++(10.4,0) ;
        \coordinate (workerAnchor) at (w\p_\j.east);
      }
      \coordinate (primaryAnchor) at (p\p.east);
    }
    %%%%%%%%%%%%%%%%%%%%%%%%%%%%%%%%%%%%%%%%%%%%%%%%%%%%%%%%%%%%%%%%%%%%%%%%%%%%%%%% 
    % a first transaction tx1
    \node (tx1) at ([xshift=3.5ex]w1_1.east) {\tx₁};
    \draw[->,double,dotted,shorten >=-2ex] (tx1.200) ++ (200:1em) -- (tx1);
    \foreach \p in {2,...,5} {
      \node (es\p_1) at ([xshift=6ex-\p ex]tx1|-w\p_1) {\es₁};
      \draw[->] (tx1) -- (es\p_1) ;
    }
    % a bunch of transactions tx2-5 that will make it into WHs actually
    \node (tx2) at ([xshift=3*3.5ex]w3_2.east) {\tx₂};
    \draw[->,double,dotted,shorten >=-2ex] (tx2.200) ++ (200:1em) -- (tx2);
    \foreach \p in {1,2,4,5} {
      \node (es\p_2) at ([xshift=6ex-\p ex]tx2|-w\p_2) {\es₂};
      \draw[->] (tx2) -- (es\p_2) ;
    }
    \node (tx3) at ([xshift=5*3.5ex]w3_2.east) {\tx₃};
    \draw[->,double,dotted,shorten >=-2ex] (tx3.200) ++ (200:1em) -- (tx3);
    \foreach \p in {1,2,4,5} {
      \node (es\p_3) at ([xshift=6ex-\p ex]tx3|-w\p_2) {\es₃};
      \draw[->] (tx3) -- (es\p_3) ;
    }
    \node (tx4) at ([xshift=7*3.5ex]w3_1.east) {\tx₄};
    \draw[->,double,dotted,shorten >=-2ex] (tx4.200) ++ (200:1em) -- (tx4);
    \foreach \p in {1,2,4,5} {
      \node (es\p_4) at ([xshift=6ex-\p ex]tx4|-w\p_1) {\es₄};
      \draw[->] (tx4) -- (es\p_4) ;
    }
    \node (tx5) at ([xshift=9*3.5ex]w3_1.east) {\tx₅};
    \draw[->,double,dotted,shorten >=-2ex] (tx5.200) ++ (200:1em) -- (tx5);
    \foreach \p in {1,2,4,5} {
      \node (es\p_5) at ([xshift=6ex-\p ex]tx5|-w\p_1) {\es₅};
      \draw[->] (tx5) -- (es\p_5) ;
    }
    % collection of txs into worker hashes wh1 and wh2
    \node[right=.1ex of tx5] (wh1) {\wh₁};
    \draw[dotted,bend right=2.5ex] (wh1.center) to (tx5.center);
    \draw[dotted,bend right=5ex] (wh1.center) to (tx4.center);
    \node[right=3.5ex of wh1|-tx3] (wh2) {\wh₂};
    \draw[dotted,bend left=2.5ex] (wh2.center) to (tx3.center);
    \draw[dotted,bend left=5ex] (wh2.center) to (tx2.center);
    %%%%%%%%%%%%%%%%%%%%%%%%%%%%%%%%%%%%%%%%%%%%%%%%%%%%%%%%%%%%%%%%%%%%%%%%%%%%%%%%
    % "upload" of worker hashes wh1 and wh2 to the primary, wh1' and wh2'
    \node (wh1') at ([xshift=-2ex]wh2|-p3) {\wh₁};
    \draw[->] (wh1) -- (wh1');
    \node[right=0ex of wh1'] (wh2') {\wh₂};
    \draw[->] (wh2) -- (wh2');
    % the following should be much later, but ... (single harmless hack)
    \node[right=1ex of wh2',fill=white] (hd3) {\hd};
    \foreach \k in {1,2}
    \draw[dotted,bend right=9ex-3*\k ex] (hd3) to (wh\k'.center);
    %%%%%%%%%%%%%%%%%%%%%%%%%%%%%%%%%%%%%%%%%%%%%%%%%%%%%%%%%%%%%%%%%%%%%%%%%%%%%%%%
    % dissemination of worker hashes wh1 and wh2
    \foreach \p in {1,2,4,5} {
      \node (wh1_\p_1) at ([xshift=3ex-\p ex]wh1'|-w\p_1) {\wh[]₁};
      \draw[->] (wh1) -- (wh1_\p_1) ;
    }
    \foreach \p in {1,2,4,5} {
      \node (wh2_\p_2) at ([xshift=6ex-\p ex]wh2'|-w\p_2) {\wh[]₂};
      \draw[->] (wh2) -- (wh2_\p_2) ;
    }
    %%%%%%%%%%%%%%%%%%%%%%%%%%%%%%%%%%%%%%%%%%%%%%%%%%%%%%%%%%%%%%%%%%%%%%%%%%%%%%%%
    % worker hash upload at receiving validators
    \foreach \whx in {1,2} { % for each of the two worker hashes
      \foreach \p in {1,2,4,5} { % for each "other" validator
        \node (wh\whx_\p_\whx') at ([xshift=4ex-\whx ex]wh\whx_\p_\whx|-p\p) {\wh[]\ensuremath{{}_\whx}};
        \draw[->] (wh\whx_\p_\whx) -- (wh\whx_\p_\whx');
        % and header creation (after second block)
        % ... and sending availability votes
        \ifthenelse{\equal{\whx}{2}}%
        {\node[right=2ex of wh\whx_\p_\whx',fill=white,draw=none] (hd\p) {\hd}; 
          \foreach \k in {1,2}
          \draw[dotted,bend right=9ex-3*\k ex] (hd\p) to (wh\k_\p_\k'.center);
        }%
        {}%
      }
    }
    %%%%%%%%%%%%%%%%%%%%%%%%%%%%%%%%%%%%%%%%%%%%%%%%%%%%%%%%%%%%%%%%%%%%%%%%%%%%%%%%
    % collect "foreign" whs into availability votes for a block
    \coordinate (ac) at ([xshift=2ex,yshift=2ex]hd1|-p3);
    \foreach \p in {1,2,4} { % for each "other" validator
      \node[fill=white] (av\p) at (ac){\hd\makebox[0pt][l]{\ensuremath{{}_{\sim\p}}}};
      \coordinate (ac) at ([xshift=-1ex,yshift=-1ex]ac);
    }
    \foreach \p in {1,2,4} { % for each "other" validator
      \draw[->] (hd\p) -- (av\p.east);
    }
    \node[fill=white] (av3) at (ac){\hd\makebox[0pt][l]{\ensuremath{{}_{\sim3}}}};
    \draw[->] (hd3) -- (av3);
    \begin{pgfonlayer}{background}
      \node[fit={(av3)(av1)([xshift=2ex]av1.north east)},fill=lightgray] (ac3) {};
    \end{pgfonlayer}
    % broadcast the availability certificate
    \foreach \p in {1,...,5} { % for each "other" validator
      \node (ac\p') at ([xshift=17ex-\p ex]ac|-p\p) {\ac\ensuremath{{}_\p}};
      \draw[->] ([yshift=2.5ex-\p ex]ac3.east) -- (ac\p'); 
    }
  \end{tikzpicture}%
  }
  \caption{The availability protocol in the genesis round}
  \label{fig:availability-protocol}
\end{figure}
%\end{comment}
\section{Primary actions}


\subsection{Availability at genesis}
The pure availability protocol: genesis actions

\begin{description}
\item[Genesis header compilation ({\hd[]})] 
  \tnote{[primary]}
  If a primary has obtained a complete set of 
  worker hashes for the genesis round
  from its workers, 
  it can compile a block header. 
  The process of header compilation 
  does not distinguish between worker hashes of its local workers
  and those with transactions that were received at other validators. 
  At genesis,
  the header consists of the creator's identity 
  and the list of worker hashes.
  
\item[Availability voting/commitment (\hdₚ→)]
  \tnote{primary\\ → primary}
  A genesis header of a primary is acceptable 
  if all its worker hashes have been forwarded
  by the local workers
  (which are trusted to have checked these worker hashes).
  The latter implies that the relevant erasure coding shares
  are kept available. 
  An availability commitment is made
  by signing the header 
  and sending the signed header to the creator
  (for the purpose of aggregation into availability certificates). 

\item[Commitment aggregation (\hdₚ←)]
  \tnote{[primary]}
  The signatures of received availability commitments 
  are aggregated into certificates of availability 
  for a header. 

\item[Certificate broadcast (\ac⇒)]
  \tnote{primary\\ ⇒ primary}
  Once a primary receives commitments
  from a global weak quorum for its genesis header,
  it broadcasts the certificate of availability.  

\item[optional header distribution] 
  ~\todo[inline]{%
    Right now,
    there seems no reason to add this ``safe guard'';
    it might be confusing.
  }
  One might expect that header creators send their headers around.
  However,
  there is no need for this.%
  \endnote{... at least in theory, we'll keep it slick for the moment} 
  
  % has uploaded a worker hash 
  % for the genesis round, 
  % optionally, 
  % the primary \emph{can} distribute 
  % the genesis header:
  % ‼{\color{red} why do we do this then at all?}
  % \begin{itemize}
  % \item the primary~\(p\)
  % \item worker hashes \(\wh_1 \cdots \wh_n\), 
  %   produces by the local workers at \(p\)'s validator. 
  % \end{itemize}
  % In theory,
  % this step is not necessary,
  % since each other primary will eventually receive
  % the  worker hashes \(\wh_1 \cdots \wh_n\),  
  % forwarded from it's workers. 
  
  % Note,
  % the creating primary of the header does not sign it;
  % the signature is “delayed” until 
  % enough primaries have commited to storing 
  % the underlying data,
  % as described by the next action.

  % ‼[a “sequence number” with the set of “target” learners should suffice,
  % to trigger such an header request]
\end{description}

A (partial) execution of the availability protocol at genesis is
illustrated in \fig\ref{fig:availability-protocol}.

\FloatBarrier

\subsection{Integrity: the general case at once}
First off,
the integrity-protocol
re-uses the sending of signed headers~\(\hdₚ\) to the creator 
(from the availability-protocol), 
as a commitment of the signer to 
one unique header for the  validator (and round),  
namely the first one signed and sent. 
Thus, 
correct validators will not sign and send any other header 
for the respective  creator of the header (for the same round). 


\begin{description}
\item[Integrity signing \(\hdₚ\)]
  \tnote{primary ⇒ primary}
  Signing and sending the header to the creator implies that 
  (a correct) primary will not sign any other header
  of the same creator with the same round number. 

\item[Block aggregation (\bk)]
  \tnote{[primary]}
  Using 
  the same signature aggregation mechanism that
  is used for availability certificates, 
  validators will aggregate additional signatures 
  (besides the availability signature) for their block headers,
  producing learner-specific \emph{blocks}, 
  which, by definition, 
  are headers signed by a learner-specific quorum. 

\item[Block broadcast \(\bk\)] 
  \tnote{primary ⇒ primary}
  The aggregated signatures of a block header form 
  a \emph{learner-specific block}. 
  The signature aggreator will broadcast 
  to all primaries that belong to some quorum of the respective learner. 
  (Later,
  these will be used as references to 
  previous blocks in the learner-specific \Dag[s]
  see \ref{???}) %FIXME reference.
\end{description}

There is no conceptual difference between 
the integrity protocol at genesis,
comparted to the typical case. 
The only difference is that 
headers in the general case 
will carry additional information. 
Thus,
we can finish the description of the protocol, 
by filling in the additional data and steps in the typical phase
of the availability-protocol 
(see also \fig\ref{fig:data-structures},
for the difference between headers at genesis and the typical phase). 

\begin{figure}[htb]
  \centering
  \tikzstyle{every node}+=[outer sep=0pt,inner sep=1pt]
  \newcommand{\primaryDistance}{15ex}
  \newcommand{\workerPrimaryDistance}{1ex}
  \newcommand{\workerDistance}{3ex}
  \scalebox{.8}{%
    \footnotesize%
    \begin{tikzpicture}[scale=1,thick]
      %%%%%%%%%%%%%%%%%%%%%%%%%%%%%%%%%%%%%%%%%%%%%%%%%%%%%%%%%%%%%%%%%%%%%%%%%%%%%%%% 
      % The message passing diagram of the integrity protocol
      %%%%%%%%%%%%%%%%%%%%%%%%%%%%%%%%%%%%%%%%%%%%%%%%%%%%%%%%%%%%%%%%%%%%%%%%%%% 
      \coordinate (primaryAnchor) at (0,0);
      \foreach \p in {1,...,7} {
        \node[below=\primaryDistance of primaryAnchor,anchor=east] (p\p) 
        at (primaryAnchor) {\ensuremath{\text{primary}_\p}};
        \draw[->] (p\p) -- ++(15,0);
        \coordinate (primaryAnchor) at (p\p.east);
      }
      \begin{pgfonlayer}{background}
        \foreach \p in {1,...,4}{
          \node[pattern=crosshatch, pattern color=green!80!black,fit={(p\p)},inner sep=1.5ex] {};
          \node[fill=white, fit={(p\p)}] {};
        }
        \foreach \p in {1,3,5,7}{
          \fill[pattern=horizontal lines, pattern color=blue]
          (p\p.north east) -- (p\p.south west) -- (p\p.south east) -- cycle;
        }
        \foreach \p in {3,...,7}{
          \fill[pattern=vertical lines, pattern color=blue]
          (p\p.north east) -- (p\p.south west) -- (p\p.north west) -- cycle;
        }
      \end{pgfonlayer}

      \node[inner sep = 4ex] (theOrigin) at ([xshift=-3ex]p3.10) {};
      \foreach \p in {1,...,7} { % for each "other" validator
        \coordinate (ac\p') at ([xshift=17ex-\p ex]theOrigin|-p\p) ;
    }
    \node (ac3') at ([xshift=17ex-3 ex]theOrigin|-p3) {\ac\ensuremath{{}_3}};
    \draw[->] (theOrigin) -- ([xshift=4ex,yshift=4ex]ac3'); 
    \coordinate[right=14ex of ac3'] (sigAgg);
    \foreach \p in {1,2,4} { % for each "other" validator
      \node[fill=white] (av\p) at (sigAgg){\hd\makebox[0pt][l]{\ensuremath{{}_{\sim\p}}}};
      \coordinate (sigAgg) at ([xshift=-1ex,yshift=-1ex]sigAgg);
    }
    \foreach \p/\y in {1/2,2/1,4/-14} {
      \node[left=\y ex of ac\p',fill=white] (hd\p') {\hd};
      \draw[->] (hd\p') -- (av\p);
    }
    \begin{pgfonlayer}{background}
      
      \node[fit={(av4)([xshift=2ex]av1.north east)([xshift=-2ex]ac3'.west)},pattern=crosshatch,pattern color=green!80!black,draw] (bkGreen) {};
    \end{pgfonlayer}
    \foreach \p in {1,...,4} { % for each "green" validator
      \node[right=25 ex of ac\p'] (bkGreen\p') {\bk};
      \draw[->] (bkGreen) -- (bkGreen\p'); 
    }
    \foreach \p in {5,...,7} {
      \node[right=35 ex of ac\p',fill=white] (hd\p') {\hd};
      \draw[->] (hd\p') -- (ac\p');
    }
    \coordinate[right=15 ex of bkGreen3'] (sigAgg);
    \foreach \p in {7,...,5} { % for each "other" validator
      \node[fill=white] (av\p) at (sigAgg){\hd\makebox[0pt][l]{\ensuremath{{}_{\sim\p}}}};
      \draw[->] (hd\p') -- ([xshift=1ex,yshift=-.5ex]av\p.south east);
      \coordinate (sigAgg) at ([xshift=-1ex,yshift=-1ex]sigAgg);
    }
      \begin{pgfonlayer}{background}
        
        \node[fit={(bkGreen3')([xshift=2ex]av7.north east)(av5)},pattern=vertical lines,pattern color=blue,draw] (bkBlue) {};
      \end{pgfonlayer}
    \foreach \p in {7,...,3,1} { % for each validator in some "blue" quorum, 
      \node[right=65 ex of ac\p'] (bkBlue\p') {\bk[blue]\ensuremath{{}_3}};
      \draw[->] (bkBlue.east) -- (bkBlue\p'); 
    }
    \coordinate (top) at (bkBlue3'.east|-p1); 
    \coordinate (bottom) at (bkBlue3'.east|-p7); 
    \draw[double,very thick,dotted] ([xshift=1ex,yshift=1ex]top) -- ([xshift=1ex,yshift=-1ex]bottom);
    \coordinate (sigAgg) at ([xshift=20ex]bkBlue5');
    \foreach \p in {7,...,4} { % for each "other" blue validator, not 3
      \node[fill=white] (bk\p) at (sigAgg){\bk[blue]\makebox[0pt][l]{\ensuremath{{}_\p}}};
      %\draw[->] (bk\p') -- ([xshift=1ex,yshift=-.5ex]av\p.south east);
      \coordinate (sigAgg) at ([xshift=-1ex,yshift=-1ex]sigAgg);
      \draw[->] (sigAgg|-p\p) -- (bk\p.south east);
    }    
    \begin{pgfonlayer}{background}
      \node[fit={(bkBlue5')([xshift=2ex]bk7.north east)(bk4.south west)},fill=lightgray] (BlueBlocks) {};
    \end{pgfonlayer}
    \foreach \p in {3,5,6,7} {% for some blue learners
      \node[draw=blue] (qs\p) at ([xshift=5ex]BlueBlocks.east|-p\p) {\qs[5]};
      \draw (BlueBlocks.east) -- (qs\p);
    }
  \end{tikzpicture}%
    \todo{for cooler patternage
      \url{https://tex.stackexchange.com/questions/597172/tikz-set-the-line-width-of-the-pattern}
    }
  }
  \caption[Integrity protocol]{%
    The integrity protocol 
    (concluding each round that's was “opened” in the availability-protocol)%
  }
  % 
  \label{fig:integrity-protocol}
\end{figure}
\todo[inline]{%
  blue blocks go to all validators in 
  %\begin{enumerate}
  % \item
  ① the signing quorum
  % \item
  ② all validators that are in some blue quorum 
  %\end{enumerate}
}

\subsection{Availability: the typical case}

\begin{description}
\item[Generating and broadcasting signed quorums]
  \tnote{primary \\⇒ primary}
  Once a validator has collected 
  enough new  blocks (for a learner), 
  it signs a learner-specific quorum of such blocks;
  the result is called a \emph{signed quorum},  
  for short. 
  All these blocks have to be from the same round. {\color{red} important ‼}

  Under certain conditions,
  in particular if there is exceptional delay for a specific learner,
  one can forgoe announcing a proper signed quorum
  and instead signs a \emph{dummy quorum} for a specific learner,
  \ie a signature over the ID of the learner in question and the current round number.  

\item[General header compilation]
  The biggest additional work and data
  concerns the compilation of headers.
  In the typical phase, 
  a header carries two additional data items, namely
  \begin{itemize}
  \item 
    the availability certificate of the previous header 
    of the header's creator/initiator
  \item 
    hashes of the signed (dummy) quorums sent by the same validator
  \end{itemize}

\item[General header checking]
  As signed quorums also serve as certificates of availability,
  checking a signed quorum amounts to checking the signed certificates.\endnote{%
    Somehow it seems overkill to have (hases of) signed quorums in the headers.   }
  In a similar way,
  the certificate of availability amounts to a checking of signatures.
\end{description}

\todo[inline]{describe in detail how 

  the checking of the availability of the headers takes place}

\subsection{Summary}
The availability protocol in a non-genesis round 
only differs in having
\begin{enumerate}
\item the additional requirement 
that each block header also includes 
the certificate of availability 
for the previous header of the same validator and 
\item 
the sending and checking of signed quorums
(each of which implements the reference to 
blocks from the previous round—in a learner-specific \Dag).
\end{enumerate}

As a consequence,
casting an availability vote / sending a commitment message 
\todo{discuss terminnolgy}
becomes a recursive commitment
to storing all blocks until genesis 
(or the last block that some of learners might still want availabl). 

\section{Data structures}

\begin{figure}[htb]
  \centering
% \subfloat[\color{violet} \bf Missing]{
%   \begin{minipage}{.3\linewidth}
%       \begin{itemize}
%   \item Integrity Vote (cf. Availability Vote → Storing promise ?)
%   \item ~ 
%   \end{itemize}
%   \end{minipage}
% }

  \subfloat[Transaction received by worker~\(w\)]{
    \tx:
    \begin{tikzpicture}[baseline={([yshift=-.5ex]b.center)}]
      \node[ellipse,fill=black] (b){
        \textcolor{white}{\bf\footnotesize\begin{tabular}[c]{c}
                                            trasaction\\
          data
        \end{tabular}}
    };
    \node[anchor=west] (w) at (b.south east) {\small\({}\to w\)};
    \end{tikzpicture}
  }
  \qquad
  \subfloat[Transaction copy (trivial erasure share)]{
    \es: \tx
    % \begin{tikzpicture}[baseline={([yshift=-.5ex]b.center)}]
    %   \phantom{
    %   \node[ellipse,fill=black] (b){
    %     \textcolor{white}{\bf\footnotesize\begin{tabular}[c]{c}
    %       blob\\
    %       of\\
    %       data
    %     \end{tabular}}
    %   };}
    % % https://texample.net/tikz/examples/torn-paper/ 
    % % ‼ make cute torn edges
    % \clip[fill] ([xshift=-1em]b.north east)
    % -- ([yshift=1em]b.south west) 
    % -- ([xshift=1em]b.south west)
    % -- ([yshift=-1em]b.north east) -- cycle;
    % \node[ellipse,fill=black] (b){
    %     \textcolor{white}{\bf\footnotesize\begin{tabular}[c]{c}
    %       blob\\
    %       of\\
    %       data
    %     \end{tabular}}
    %   };
    % \end{tikzpicture}
}
  % \qquad
  % \subfloat[Worker \(y\) is commiting to the hash of~\(x\)]{
  %   \(y♯x\): \([\#(x)]_{\sim y}\)
  % }%‼

\subfloat[Batch hash of worker~\(w\)]{
  \#(\(\overrightarrow\tx\)):
  \#
    \(\left(\begin{tikzpicture}[baseline={(batch.center)}]
      \node[anchor=west] (batch) {
        \(\colorbox{lightgray}{\(\begin{array}[c]{c}
          {\tx}\\%_{\to w}
          {\tx}\\%_{\to w}
          {\tx}\\%_{\to w}
          {\tx}\\%_{\to w}
          {\tx}\\%_{\to w}
          {\tx}%_{\to w}
        \end{array}\)}\)
    };
      \end{tikzpicture}\right)\)
  }
  \subfloat[%
  Worker hash (issued by~\(w\)), 
  including the round number \(\rnd\),
  and the number of transactions~\(\|\protect\overrightarrow \tx\|\);
  it is signed by~\(w\)]{
    \wh:
    \begin{tabular}[t]{@{}l@{}}
      \begin{tikzpicture}[baseline={([yshift=-.5ex]wh.center)}]
        \node (wh){\(\left[ \#(\overrightarrow \tx), \rnd,  \|\overrightarrow \tx\| \right]_{w}\)}; 
      \end{tikzpicture}
      %\footnotesize
    % {\color{violet} + info for correctness checking} \\
    % (\eg number of \tx[s], or list of \#s)\\
    %   \emph{Tahoe – The Least-Authority Filesystem}
    \end{tabular}
  }

  \subfloat[Genesis Header]{
    \(\hd[]\):\(
\tikz[baseline={(x.base)}]{\node (x){
\(\left(p,\overrightarrow\wh\right)\)
};}\)
  }
\qquad
  \subfloat[Genesis certificate of availability]{
    \(\ac\): 
    \(\Bigl[\hd[]\Bigr]_{\overrightarrow q}\)
  }

  \subfloat[Block]{
\bk:
    \begin{math}
      \left[\hd\right]_{\color{green!60!black}{\sim p_1 \dots \sim p_m} [ \color{green!60!black}\sim p_{m+1} \cdots \sim p_{k}]}
    \end{math}
  }
  \qquad
  \subfloat[Signed quorum]{
    \(\qs_p\):
    \begin{math}
      \begin{array}[c]{@{\rhd}l}
        [\bk_1 \cdots \bk_\ell]_{\sim p}
      \end{array}
    \end{math}
  }

  \subfloat[Header]{
    \(\hd\):\(
\tikz[baseline={(x.base)}]{\node (x){
\(\left(p,\overrightarrow\wh,\ac, \overrightarrow {\#(\qsₚ)}\right)\)
};}\)
  }


  \caption{Overview of data structures}
  \label{fig:data-structures}

\end{figure}
\FloatBarrier






\bibliographystyle{alpha}
\bibliography{HN.bib}


\appendix
\printendnotes


\section{food for thought}
\subsection{new}



\subsubsection{some of Tobias' brainstorming}
how can we make chimera chains safer?

\begin{description}
\item[Requirement candidates]~\\
  \begin{itemize}
  \item
    \emph{absolutism:}
    if 
    a transfer between two base ledgers is commited in 
    a “healthy” chimera chain, 
    then the transfer is also (essentially) commited in both ledgers
    \endnote{%
      Here, the question arises,
      if a potential roll-back in case of 
    },
    i.e., eventually it will go through, unless we've got a “liveness fault”
    \todo{whatever a liveness fault is}
  \item 
    \emph{justified absolutism:}
    each consensus on an anchor block selection in 
    a chimera chain needs “double consensus” from both chains,
    i.e., 
    effectiverly,
    both chains need to approve the anchor block,
    which might lead to liveness issues ? 
    
  \item
    \emph{absolutism grounding:}
    does each anchor block in the chimera chain
    have also references to blocks in the underlying chain, 
    “super-supporting” the anchor?

  \item
    \emph{user authorization:}
    users may choose to effectively use
    the chimera chain as a limited bridge,
    “temporariliy” moving (some of their accounts)
    to the chimera chain
    \todo{%
      what's the difference to the bridge then here?%
    }
  \item
    \emph{user's ruling:}
    moving assets into other chains
    -- e.g., chimera but not necessarily --
    for atomic settlement on the other chain

    
  \end{itemize}
\end{description}


\subsection{old}

\subsubsection{}


\begin{itemize}
\item
  lots of state is completely independent of each other, 
  can we use this for optimization using this ``concurrency''
\item 
  how does this state right tool work ? 
\item 
  How do the message graphs of HP
  compare to Narwhal DAGs
\end{itemize}
\paragraph{Isaac's thoughts}

\begin{itemize}
\item message sharing (on the primary level) might be slow
\item 
\end{itemize}

\href{%
https://github.com/anoma/specs/blob/main/src/architecture/consensus/typhon/mempool.md%
}{the specs}

\subsection{very old, but ... very researchy}

\paragraph{is reliable broadcast in narwhal actually even more useful?}
does Narwhal make consensus easier
in that it restricts the set of valid values to propose?
In particular,
proposals of non-existing blocks are impossible?
cf. ideas in
\href{https://www.sciencedirect.com/science/article/pii/089054018790054X}{Asynchronous Byzantine agreement protocols}


\paragraph{what can be gained by randomness? at what price?}

well, just for the record,
this is an interesting venue for (much) more performance,
which is probably much harder to analyze ... 
\begin{itemize}
\item \href{https://link.springer.com/article/10.1007/s10009-020-00603-x}{Verification of randomized consensus algorithms under round-rigid adversaries}
\item \emph{Randomized Byzantine Generals} Michael O.~Rabin,  1983
\end{itemize}





\section{specific questions}
\begin{itemize}
\item 
  “availability certificate Availability Certificate: an aggregation of signatures from a Weak Quorum attesting that everything referenced by a particular Header is available. \bf Must include a signature from the Header's primary.”
  \begin{itemize}
  \item
\begin{verbatim}
Where is this signature?
\end{verbatim}
  \item
\begin{verbatim}
What does it sign?
\end{verbatim}




  \end{itemize}
\item my acknowledgment, “Availability Vote” --
\begin{verbatim}
isn't it rather a **promise/pledge**
\end{verbatim}
\item about ``blocks'' 
\begin{verbatim}
does it make sense to call these learner blocks, possibly even certified learner blocks?
\end{verbatim}
\item about \tx{s}: 
\begin{verbatim}
  a learner might actually ignore a big chuck of transactions?
\end{verbatim}
\item about the ``two'' protocols
\begin{verbatim}
 if the integrity protocol gets stuck, 
 the availability protocol will stuck?
\end{verbatim}
yes
\begin{verbatim}
   also vice versa ? 
\end{verbatim}
partially 

\item more about blocks
\begin{verbatim}
headers have "sequence numbers" (stemming from the creator)
\end{verbatim}

\item about signed quorum
\begin{verbatim}
would it make sense to call these 
signed_certified learner blocks_
\end{verbatim}
  


\begin{verbatim}
WHEN ARE BLOCKS broadcast and by whom?! 
\end{verbatim}
\color{green!60!black}
it *is* a broadcast and it is performed
\begin{itemize}
\item after completion (in analogy to availability certificates)
\item by the block creator 
\end{itemize}

\end{itemize}

---
\section{Random snippets}

\begin{itemize}
\item Besides the learner graph,
we assume that each transaction is relevant only to a subset of the learners.%
\endnote{%
  \emph{Does it make sense to have transactions that no learner subscribes to?}
  Probaly not! Garbage collection could be triggered by 
}
\ 
\item 
\end{itemize}

---

\begin{description}
\item[New header construction and broadcast] 
  \tnote{primary\\ ⇒ primary}
  The following conditions will trigger the production of a new header.

  \begin{itemize}
  \item
    Each worker on the primary's validator has provided a
    worker has for the validators current round. 
  \item If not genesis round,
    ... %‼ certificate of availability of it's own previous header / block
  \item 
    If not genesis round,
    ...%‼ signed quorum 
  \end{itemize}

  Once, 
  the header is constructed \textcolor{violet}{(but NOT signed?)}, %‼ 
  the primary broadcasts it to all other primaries. 

\item[Header checking and acknowledgment (Availability Vote)]
  A received header of a primary is checked, 
  according to the following check list. 
  
  \begin{itemize}
  \item
    The worker hashes of the header must have been “uploaded”
    by the local workers
    (which are trusted to have check theses worker hashes).
    
  \item If not genesis round,
    ... %‼ certificate of availability of it's own previous header / block
  \item 
    If not genesis round,
    ...%‼ signed quorum 
  \end{itemize}

  After checking all this,
  the primary sents an acknowledment back to the header producer, 
  \ie a message with the header signed. 


\item[Availability Certificate generation and broadcast]
  \tnote{primary\\ ⇒ ∀primary}
  After receiving a global weak quorum of header acknowledments 
  for a previously broadcast header, 
  the received signatures are aggreagated
  \textcolor{violet}{\bf and singed}. %‼ 

  The result is broadcast to all primaries. 

\item[Waiting for (certified learner) blocks]
  \fbox{\color{violet}confusing switch to integrity protocol!}

\item[Announcing signed quorums]
  Once a validator has enough 
  “fresh” blocks (see below) for a learner from a “preceeding” round, 
  the primary signs a learner-specific quorum of such blocks.
  However,
  under certain conditions,
  it might be useful to anounce \emph{empty} signed quorums, 
  indicating that the next block header will not include 
  any signed quorums for a specific learner. 
\end{description}

\subsection{Primary actions in the integrity protocol}
\label{sec:prim-acti-integr}

\begin{enumerate}

\item[Uniqueness attestation / Integrity Vote]
  \tnote{primary\\ → primary}  
  When a primary receives a header from another validator
  with a round number that directly succeeds 
  the last known header of the sending validator, 
  ‼[(for a new \emph{“round”})] %
  the primary signs that header (additionally for the purpose of integerity)
  and sends it back to the creator of the header.

  \color{violet}\item[Certified learner blocks (Integrity certificate for headers)]
  \color{violet}
  \tnote{primary\\ ⇒ primary}
  Upon receiving a learner-specific quorum of integriy votes for a header, 
  the primary aggregates these
  and broadcasts a certied learner block. 

\item[Signed block quorums]
  
  
  


\end{enumerate}


\subsubsection{Genesis round}

\begin{description}
\item[Header compilation]
  \tnote{primary → primary}
  Whenever a full set of worker hashes from another validator
  has ,
  the primary 
  
\end{description}

---

\paragraph{learner-specific round numbers}

\begin{itemize}
\item Each learner might “observe” different round numbers. 
\item In first approximation: do not expect any synchrony whatsoever!
\end{itemize}


\paragraph{reference a quorum of blocks from the previous round}
This is seen as follows,
(for non-genesis blocks):

\begin{itemize}
\item the block contains a header \hd
\item 
\end{itemize}

\paragraph{learner-specific DAG structure}

\section{Erasure coding}
The worker that has received a transaction from a client
    generates a suitable erasure code,
    broken up into a finite set of shares%
    \footnote{%
      Shares are also known as \emph{chunks}.%
    }%
    \
    to be distributed over all validators. 
    The share distribution has to be such that
    %“
    any quorum of validators (relative to any learner)%
    \endnote{Here, we want to put
    “(relative to any learner \emph{that subscribes to the transaction})”})
    can re-construct the transaction data %
    %”
    from the set of shares 
    that they obtained collectively. %
    %‼ Discuss and give specific example 
    {‼[check this:]\color{red}
    Moreover, 
    the map from 
    erasure coding shares of the transaction
    to (mirror) workers that receive the respective share
    is determined by the worker's index
    (and the identifier of its validator). 
    \todo{\tiny
      %‼ 
      cf. knowing whether we have all the shares ?
    }
    }

\section{State partition and fractal instances}
\label{sec:state-partition}
Learners,
\eg execution engines,
want to be responsible for
\emph{changes} to the smallest possible part of the state.
However,
to enable basic actions such as cross-chain transfers, 
learners have to gather enough information 
about the global state to determine whether 
a given transaction \tx is actually ‼[valid / executable]. 
The matter becomes delicate
if a transaction depends on parts of the state 
that different learners are responsible for. 
No learner can single-handedly determine if
a given commited transaction is executable, 
unless some learner is tracking the complete global state\xspace% 
---an almost impossible task already our days!
Now,
let us focus on transactions whose “inputs” (or “outputs”) are spread out 
over several learners%
\st{ as these are the trouble makers}%
.

\todo[prepend,inline, caption={On chimera chains}]{%
  
★ ``lock'' the relevant parts of the state on all involved learners
\footnote{%
  footnotes in \texttt{\textbackslash todo} notes
  have to be \texttt{\textbackslash protect}ed, 
  unless we put a caption. 
}
 \\
 ★ ``spawn'' a chimera chain\\
 ★ ``move'' locked state fragments to chimera chain\\
 ★ ``settle'' on chimera chain\\
 ★ ``finalize'' chimera chain\\
 ★ ``re-import projected state''
}
% Howevever,
% the validity of a transaction can depend on 
% parts of the state, 
% which lie outside the area of primary interest. 
We assume that each transition carries enough information 
to instantly deduce which part of the global state is accessed,
possibly distinguishing between reading and writing. 

\section{Old Intro}

\paragraph{Motivation%
\endnote{%
  Paragraph 1: Motivation.
  At a high level, 
  what is the problem area you are working in and why is it important? 
  It is important to set the larger context here. 
  Why is the problem of interest and importance to 
  the larger community?
}%
}
Bridges%
\st{ suck} %
have been the source of sorrow in the past. 
Even if they do work as intended, 
at best, they connect a single pair of chains. 
Building bridges between every pair of chains
is ‼[expensive]. %
Moreover, 
we want that any number of participants,
holding assets on several \emph{different} ledgers\endnote{%
    - base chains
    - root chains
    - ledgers 
} to ‼[“interact” “directly”].
\todo{explain “interact” “directly”}
%\newcommand{\dl}[1][]{\textsc{dl}#1\xspace}(\dl[s])
\todo[inline]{
  argumentation flaw:
  if we have a base ledger count of~\(n\), 
  we will have a quadratic number of chimera chains as well,
  i.e., (at least) one learner for each chimera chain. 
}
\paragraph{Probblem statement%
  \endnote{%
    Paragraph 2:
    What is the specific problem considered in this paper? 
    This paragraph narrows down the topic area of the paper. 
    In the first paragraph you have established general context and importance.
    Here you establish specific context and background. 
  }%
}
‼[
\tt How to enable any number of actors to interact directly
with moderate ressource consumption and suitable latency, 
us-ing only a “minimal” number validators on the involved base ledgers.
%
]
The ideal solution would be 
the atomic execution of a single transaction that
makes reference to all invovled base ledgers.%
\footnote{%
  This assumes counterparty discovery 
  before the transaction  is crafted.
}%
\xspace
Can we operate a protocol
that strikes a good compromise between
the number of validators necessary to order (and execute) transactions,
not spending more ressources than complete pair-wise bridging?

{
  \Huge the above does not cut it
}

\section{Some ideas for how to explain things}

\subsubsection{The risky fast track}
\label{sec:fast-n-furious}

The opposite extreme consists in
not updating any (liveness) quorums
and just hoping for “enough” overlap 
of assumed to be live quorums:
however,
it might be the case
that the intersections do 
not hold enough honest validators.
In the extreme,
none.

\subsubsection{First assessment in view of validator overlap}
\label{sec:assessment-i}

In view of a large overlap,
note that both scenarios gain in practicability:
if every pair of vaildator quorums
has a large overlap,
the increase in communication complexity is mitigated.
Similarly,
for the fast track, 
if overlaps are of validator quorums are big enough,
we have a good chance to have at least 
one correct validator in their intersection.\endnote{%
  research question:
  can we add randomization \emph{orthogonally},
  i.e.,
  without changing the original algorithms
  of heterogeneous narwhal / paxos
}

\subsubsection{Growing the overlap}
\label{sec:growing-the-overlap}

A natural question is how one could go 
about getting a bigger overlap between two chains,
for the sake of additional safety. 
In fact,
one could imagine some sort of 
“auditing” nodes,
which make it harder for Byzantine validators
to trigger safety violations \cite[opodis20HPaxos].



\subsection{Tobias's corner}

\subsubsection{excerpting the %
  \href{https://anoma.net/blog/heterogeneous-paxos-and-multi-chain-atomic-commits/}{%
    blog post%
  }
}
\label{sec:blog-post-excerpts}

\paragraph{The intuitive goal}

combo block

\begin{verbatim}
    [rb]
    /  \ 
  [r]  [b]
   |    |
   |    |
  [r]  [b]
   |    |
   |    |
  [r]  [b]
   |    |
\end{verbatim}

s.t.

\begin{quote}
  both [chains, r\&b,] agree to [unanimously] choose or [jointly] ignore the combo block.
\end{quote}

\begin{quote}
  Fortunately, 
  when there is a lot of overlap between
  validator sets of different chains, 
  these assumptions can become very reasonable.
\end{quote}

\begin{quote}
  within \emph{same} consensus
\end{quote}
\todo{undefined}

\begin{quote}
  bisimulate
\end{quote}
\todo{clashes with bisimulation ‽}

\begin{quote}
  \textbf{It's Kind of Like a Bridge}

  In some ways, 
  our Heterogeneous Consensus could create something like 
  a trusted bridge: 
  it allows operations across multiple chains,
  under specific trust assumptions. 
  Other bridge designs tend to be based on multi-phase commits,
  so ours will be faster and will not require locking assets.
\end{quote}

\begin{quote}
  green quorum is sufficient to make green chain decide on a block without communicating with anyone else
\end{quote}
\todo{%
  now,
  a quorum suddenly seems independent of the green learner
 }

 \begin{quote}
   Note that this would not be a problem if every red quorum had an intersection (featuring an honest participant) with every green quorum: red and green could avoid deciding contradictory things.
 \end{quote}
 \todo{check: 
   there's no global weak quorum
   and heterogeneous narwhal does not help either ?
 }

 \begin{quote}
    Fortunately, we have some promising directions for solving this conundrum.
  \end{quote}
  \todo{e.g., ?}

\begin{quote}
    anyone following only blue transactions or only red transactions will not observe the fork
\end{quote}
\todo{unobserved forks: 
  sounds terrible‼
}

  \begin{quote}
  "move state" from, say, the purely blue chain to the blue-green chain when they want to do business with green smart contracts, and then back to the blue chain when they're done.
\end{quote}
\todo{
  how ?
  this sounds very much like a bridge,
  and does not solve the game theoretic problem?
}

\begin{quote}
  Each piece of state (each smart contract) needs application-specific code to decide which chain holds its lock at which time.
\end{quote}
\todo{
  how to keep track of this ``locking table'' 
}
\begin{quote}
  trusted by chains
\end{quote}

\begin{quote}
  the transactions are actually executed together
\end{quote}
\todo{
  well,
  something like all execution engines do it (eventually?)
}

\begin{quote}
  prove what messages are sent
\end{quote}
\todo{
  proving what messages are sent \emph{by whom?}
  users, validators, others, …
}
\subsubsection{On the %
  \href{https://anoma.net/blog/heterogeneous-paxos-and-multi-chain-atomic-commits/}{%
    blog post%
  }
}
\label{sec:blog-post-discussion}

\paragraph{Could we merge chains into chimera chains?}

in other words,
why would we at all want to “split” again?

\begin{verbatim}
  [r]  [b]
   |    |
   |    |
  [r]  [b]
    \  / 
    [rb]
    /  \ 
  [r]  [b]
   |    |
   |    |
  [r]  [b]
   |    |
   |    |
  [r]  [b]
   |    |
\end{verbatim}

\paragraph{random questions}

\subparagraph{not all learners equal, in yet a different way}

there are two very different type of learners:
\begin{itemize}
\item end-users: small stake, non-critical
\item execution engines of validators: high stake, critical
\end{itemize}




\subparagraph{on immediate finality}
which finality do we actually need ?

\subparagraph{drawing on eigenlayer functionality?}
can we

\subparagraph{what about dynamic graphs?}
all of
\begin{itemize}
\item validator sets  
\item assumptions 
\item learner sets
\end{itemize}
can change

cf.
\begin{quote}
   Changing Quorums
\end{quote}

\subparagraph{who can initiate chimera chains / dynamic graphs?}
\begin{description}
\item[rig up] 
\item[tear down] 
\end{description}

\subparagraph{%
  on locks, not really, but ... 
}
what about garbage collection trigger 
for logged mempools




\paragraph{open issues}
Here is an open list of topics to clarify;
in principle, 
these could be in the form of github issues\todo{ (to be discussed)}. 

\subparagraph{certificates for honest behaviour}
examples, what's missing?

\subparagraph{evidence for dishonest behaviour}
examples, what's missing?

\subparagraph{garbage collection}
\begin{itemize}
\item to do or not to do
\item how to do / how to avoid 
\end{itemize}

\subparagraph{It's gonna be better than all bridges out there ‼}
the connection to bridges has to be spelled out \emph{explicitly},
in excrutiating detail

\subparagraph{different perspectives}
the difficulty in describing---and understanding---Tʏᴘʜᴏɴ
are the different perspectives that one has to keep in mind at each point in mind
\begin{itemize}
\item learner(s)
\item validators / participants / workers
\item designer
\item attacker
\item transaction batches ‽
\item chains ‽
\item something else ?
\end{itemize}
Thus, for every sentence, 
one might get it wrong,
if it is not clear who's perspective we are taking. 


\subparagraph{what's shared batch looking like anywy?}

\subparagraph{moving state}
this is a phrase, 
which begs the questions
\begin{itemize}
\item who is moving the state?
\item how is state partitioned?
\end{itemize}

Leaving the capability to move state with the user
is also an extra task the user must be willing to perform. 

\subparagraph{“As Locks”}

\texttt{chains as holding exclusive locks on state}

this model transfers to “user's rule”





\subparagraph{one of those statements to ponder about}

\begin{quote}
  This would allow applications to "move data" to shared chains where atomic commits are possible, and then back. What we need is a specific interface indicating when such moves are safe, when atomicity in batches is preserved, and a way to easily move state from one chain to another.
\end{quote}





\subparagraph{best case scenario}
great
\begin{markdown}
  - that is a good example of quorums
  - dosn't explain why atomic commit is possible
\end{markdown}
\subparagraph{\texttt{Shared batch} of transactions}
now,
what's a shared batch anyway?
a batch of transactions in a combo block ?




\begin{itemize}
\item 
  do chimeara chains need to acquire and release “ownership” ?

  \begin{itemize}
  \item 
  \end{itemize}

\end{itemize}


\end{document}

%%% Local Variables:
%%% mode: latex
%%% TeX-master: t
%%% TeX-engine: luatex
%%% TeX-command-extra-options: "-shell-escape"
%%% End:
